\documentclass[a4paper,12pt]{report}
\usepackage{../GS6}

\begin{document}

	\def \nocredits {}
	\def \LineE {Конспект по дисциплине}
	\def \LineF {Организационно-правовые основы ИБ}

	\maketitle

%03.09.15

\subsection*{Ткачёв}

Канал утечки информации - источник+среда передачи + средство разведки

Угроза утечки информации - источник + среда передачи


\begin{verbatim}
    __________    ______________    __________________
   |          |  |              |  |                  |                      
   | Источник |=>|Среда передачи|=>|Средства разведки |                                 
   |__________|  |______________|  |__________________|                         
                                                                                 
\end{verbatim}

Кто решает, как и что защищать? 
Актуальность каналов решает комиссия, мин 3 человка, один не от объекта
	
	
Эффективная СЗИ имеет стоимость не более 25\% информации
	
Угольный микрофон
Суррогатный микрофон - конструкция, не предназначенная как микрофон, ноуспешно работающая

Электродинамичесуий микорофон. Пожарная сигнализация, 

Конденсаторный - две металлических поверхности в помещении, либо провод на стене.

Электретный

Пьезоэлектрический 



Полупроводниковый микрофон

МОП-кондентатор
	
	

% 10-сен-15

Для акустической разведки применимы микрофоны с большим коэффициентом усиления. Для повышения эффективности применяются акустические усилители, имеющие резонанс в области голосовых частот. 
Для прицельного наблюдения диаграмму направленности сужают.

\begin{itemize}
	\item	Параболический направленный микрофон. Проблемы с большим геометрическим размером, необходимостью прицеливания, при прослушке наушниками может возникнуть микрофонный эффект.
	\item	Трубчатый щелевой микрофон (жарг. флейтовый). Эффективен до длины 80см.  - Одиночный направленный микрофон.
	\item	Микрофон органного типа - идентичный щелевому, но имеет множество трубок на разные резнансные частоты, на конце пакета - параболический микрофон, с звукоприёмным микрофоном. Проблемой является шуршание трубочек, очень сложен для изготовления. Труба около 1-1.5м, диаметр 20 см. Групповой по резонаторам микрофон. По количеству микрофонов - одиночный. 
	\item	Градиентный групповой микрофон - два микрофона, развёрнутых друг против друга, с вычитанием на усилителе.
	\item	Микрофонная решётка - усиливаем сигнал с полотна микрофонов, усиливается синфазный сигнал, при прихождении звуковой волны сбоку сигнал не усиливается. Направленность параболической антенны существенно выше.
	\item	Антенная решётка из пористого материала. 
\end{itemize}



	Вибороакустические каналы.
	Приёмные устройства - виброфоны. Главное - отсутствие воздушного промежутка. Пьезоэлектрические имеют наибольшее распространение. Дают очень хороший сигнал. Стетоскопы.
	
	Передача должна производиться через твёрдый посредник. Стержень предпочтителен стальной. 
	
	Наиболее распространённый приём, дальность велика.
	
	
	Гидрофоны. Происходит переход от виброфонов на железных трубах к гидрофонам на пластиковых трубах. Тонкие радиаторы способствуют считыванию.
	Отопление, водопровод.
	
	Лучше всего работает в уз-диапазоне. 
	
	
	
	Закрытие акустических и виброакустических каналов утечки.
	Виброакустические каналы закрываются излучателями белого шума.
	
	Системы зашумления включаются на период конфиденциальных переговоров, из-за высокого уровня фонового шума.
	
	Пассивные способы звукозащиты.
	На уровень до 1.5 -1.8. м ставится деревянный щит, не касающийся стены, выше натягивается ткань, под щит на стену наклеивается пористый материал. Также, это уменьшает фоновый шум от активного зашумления. Внутренние двери - предпочтительны деревянные, с уплотнением для защиты от акустики. В защищаемых помещениях оюязателен тамбур во избежание прохождения звука + виброакустическое зашумление. Глушители на вентиляцию. 
	
	+ система зашумления в вентиляцию
	
	Кабельные сети. Свободное место в канале забивается стекловатой. Доводчики обязательны.
	
	Под полы - звукопоглощающий материал. 
	
	
	////
	Лазерный микрофон (виброфон)
	
	Защита: роллеты снаружи окон, тяжёлые шторы, зашумление.
	
	Заземление в защищаемых помещениях обязательно.
	
	
	% 140915
	Лабы и особенности
	
	ЛР 1 
	Исследование интерфейса RS-232. 
	Защита - представить осциллограмму по числу и формату интерфейсу.
	
	ЛР2 
	Рассчитать количество реальных строк
	
	ЛР3 - исследование ВЧ навязывания
	
\end{document}	