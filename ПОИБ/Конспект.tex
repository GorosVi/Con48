\documentclass[a4paper,12pt]{report}
\usepackage{../GS6}

\begin{document}

	\def \nocredits {}
	\def \LineE {Конспект по дисциплине}
	\def \LineF {Организационно-правовые основы ИБ}

	\maketitle

%03.09.15

\subsection*{Кармановский}
	Организационно-правовое и методическое обеспечение ИБ . Савков Михайличенко Кармановский

	Зачёт по гостайне

\subsection{Понятие тайны. Виды тайн}
	Государственная безопасность - система гарантий государства от угроз извне и основам конституционного строя внутри страны. Для реализации в стране существует система защищаемых законом тайн. Под тайной понимается нечто скрываемое от других, известное не всем, либо просто секретное.

	Признаки тайны:
	\begin{itemize}
	\item	Сведения доверены или известны узкому кругу лиц
	\item	Сведения не подлежат разглашению
	\item	Разглашение сведений может повлечь негативные последствия (материальный или моральный ущерб её собственнику, владельцу, пользователю или иным лицам)
	\item	На лицах, которым доверена информация, не подлежащая разглашению, лежит правовая обязанность её хранить.
	\item	За разглашение этих сведений законом установлена юридическая ответственность.
	\end{itemize}

	Основные виды тайн:
	\begin{itemize}
	\item	Государственная (определена ФЗ о гостайне). Не может составлять других тайн.
	\item	Коммерческая
	\item	Персональные Данные
	\item	Тайна личной жизни
	\item	Банковская тайна
	\item	Налоговая тайна
	\item	Врачебная тайна
	\item	Тайна усыновления
	\item	Тайна связи
	\item	Нотариальная тайна
	\item	Адвокатская тайна
	\item	Тайна страхования
	\item	Служебная тайна (темно)
	\item	Тайна голосования
	\item	Тайна исповеди
	\item	есть продолжение, порядка 40 видов
	\end{itemize}



	\subsubsection{Тайна личной жизни}
	Гарантия неприкосновенности частной жизни - запрет на сбор, хранение, использование и распространение информации о частной жизни лица без его согласия.

	Нормативные акты: Конституция, ст.23-34-25, ФЗ о ПДн, Гражданский кодекс, международные договора (всеобщая декларация прав человека, Европейская конвенция, международный пакт о гражданских политических правах)



	\subsubsection{Банковская тайна}
	К объектам относятся: тайна банковского счёта, тайна операций по банковскому счёту, тайна частной жизни клиента.

	Статья 26 ФЗ о банках и банковской деятельности. При разглашении банком сведений клиент вправе требовать возмещения убытков. Сведения могут предоставляться клиентам или их представителям, судам, счётной палате, налоговым, следственным и таможенным органам.



	\subsubsection{Врачебная тайна}
	Вся информация, касающаяся факта обращения гражданина за медицинской помощью, состояния его здоровья, диагноза, данные, полученные при обследовании и лечении, сведения о личной жизни, полученные при обследовании и лечении.

	Соблюдение врачебно тайны распространяется на всех лиц, которым данная информация стала известна. Информация является тайной вне зависимости от формы обращения человека к медработникам и результата обращения.

	Главная правовая норма - ст.61 основ РФ об охране здоровья граждан. Не рассматривается нарушением врачебной тайной обмен информацией между специалистами между специалистами в процессе лечения.


	\subsubsection{Тайна усыновления}
	В РФ охраняется ст.79 Семейного кодекса РФ, должна соблюдаться лишь по желанию усыновителей.

	Для обеспечения тайны усыновления допускается изменение места рождения, даты рождения (не более 3 мес.). Разглашение вопреки желанию усыновителя влечёт меры от штрафа до уголовной ответственности.

	ilyapopov27@gmail.com
	vk.com/aw.nimble


	След. пара - заполненное ТЗ, орг форма предприятия, характеристика, автоматизированная система. (наиболее простая - СКУД)



	% next 10.09.15

	\subsubsection{Нотариальная тайна}

	$\sim$
	Освобождение от тайны только по решению суда.


	\subsubsection{Адвокатская тайна}
	Любые сведения, связанные с оказанием адвокатских услуг. Адвокат не может быть вызван и допрошен.

	Без согласи доверителя адвокат имеет право сообщать сеедения, необходисмые для

	Тайна распространяется: факт обращения, факты и документы, сведения о доверителе, содержание правовых советов,  данных доверитею => любые сведения, связанные с оказанием юридических услуг.

	\subsubsection{Тайна страхования}
	Ст. 946 ГК РФ.
	Страховщик не врпае разглашать сведения о страхователе, выгодоприобретателе.



	\subsubsection{Служебная тайна}
	Информация с ограниченным доступом, за исключением ПДн и гостайны.
	Законодательной базы не имеет, сейчас подгоняется под другие виды тайн. Гриф ДСП.



	\subsubsection{ПДн}
	\begin{itemize}
	\item	ФЗ о ПДн
	\item	Постановление прав-ва положение об обеспечении безопасности ПДн при обработке в вычислительных сетях
	\item	Приказы ФСБ, мининформсвязи, ФСТЭК
	\item	Методические документы ФСБ РФ по защите ПДн
	\item	Методические документы ФСТЭК
	\end{itemize}

	Тайна голосования

	\subsubsection{Тайна исповеди}
	Охраняемая законом тайна, п.7 ст 13 закона о свободе совести и религиозных объединениях.
	Священнослужитель не может быть допрошен в качестве свидетеля, об информации, полученной из исповеди.

	\subsection{Классификация методов защиты}
	\begin{itemize}
	\item	Технические
	\item	Программные
	\item	Криптографические
	\item	Организационные
	\end{itemize}

	По виду решаемых задач
	\begin{itemize}
	\item	Резервирование
	\item	Введение избыточности
	\item	Регулирование доступа
	\item	Регулирование использования
	\item	Защитные преобразования (кодирование, шифрование)
	\item	Контроль
	\item	Регистрация
	\item	Уничтожение
	\item	Сигнализация и реагирование
	\end{itemize}


	По функциональному назначению
	\begin{itemize}
	\item	Самостоятельные решения средств защиты
	\item	Комплексные с другими средствами решения
	\item	Средства управления средствавми защиты
	\item	Обеспечение функцционирования средсвтв защиты
	\end{itemize}

	Организационные методы защиты используются как самостоятельно,
	Организационно-технические и организационно-правовые мероприятия.

	Играют особую роль по следующим причинам:
	\begin{itemize}
	\item	Повышенное влияние случайных факторов
	\item	Неформальный характер организационных методов
	\item	Наличие человеческого фактора
	\end{itemize}


	% next-Гостайна

	Введение+ТЗ, печатно.
	Что такое орг-прав обесп без, опрделение предмета.
	Задачи : исследование де

	Аргусментация обседования скуд человеческим фактором.


	Цели Задачи Описание предприятия, АС и функционала



	%\next	17_sep.15
	\subparagraph{Основные понятия гостайны}
	Через Чт пар нет.

	Закон о гостайне. (1993), последняя поправка 8 ноя 2011.
	Гостайна - сведения политического, экономического, военного, и научно-технического характера, утрата или разглашение которых создаёт угрозу безопасности и независимости государства или наносит ущерб его интересам.
	Круг вопросов определён законом.

	Объём засекреченных сведений должен быть обоснован и минимален.
	Действие закона распространяется на территорию РФ, включая учреждения РФ за рубежом (посольства итп) и за её пределами (нет права продажи изделий с грифом секретности). Обязателен для выполнения должностными лицами и гражданами, осведомлёнными с содержанием тайны, в том числе и при выездах за границу в служебные командировки или по личным делам.

	В силу ограничительного действия закона распространён на сравнительно небольшую часть населения, \textbf{добровольно} вступившую с государством в правоотношения по защите гостайны.

	\subsubsection{Основные понятия}
	Носители сведений, составляющих гостайну: материальные объекты, в том числе физические поля,  в которых сведения, составляющие гостайну находят своё отображение в виде символов, образов, сигналов, технических решений и процессов.

	Человек не является носителем сведений, составляющих гостайну. (рассматривается в качестве субъекта правоотношений)

	Допуск к гостайне - процедура оформления права граждан на доступ к сведениям, составляющим гостайну, а предприятий, учреждений и организаций - на проведение работ с использованием таких сведений.

	Доступ к гостайне - лица, допущенные на работу к гостайне получают разрешение на работу с определёнными сведениями.
	Доступ - санкционированное полномочным должностным лицом ознакомление конкретного лица со сведениями, составляющими гостайну.

	Степень секретности определяется степенью ущерба экономического, политического, военного и др., наносимого в результате утери закрытой информации или передачи её другим лицам, организациям и государствам.

	Гриф секретности - реквизит, свидетельствующие о степени секретности сведений, содержащихся в их носителе, проставляемые на самом носителе, или в сопроводительной документации на него. Гриф содержит степень, дату засекречивания, кем засекречены, на какой срок засекречены (обязателен), основание для засекречивания. По окончании срока - документ либо рассекречивается, либо уничтожается.

	Перечень сведений, составляющих гостайну - совокупность категорий сведений, в соответствии с которыми сведения относятся к гостайне и засекречиваются. Три перечня.

	\subsubsection{Режим секретности}
	РС - установленный нормами права единый порядок обращения со сведениями, составляющими служебную и гостайны в целях предотвращения утечки закрытой информации по различным каналам.

	Особенности режима секретности
	\begin{enumerate}
	\item	Единый
	\item	Обязательный для всех госорганов, должностных лиц, предприятий, учреждений и организаций порядок обращения с государственными секретами.
	\end{enumerate}

	Персональная ответственность руководителей всех рангов за организацию режима секретности в их учреждениях ,организациях и на предприятиях за проведение необходимого комплекса мероприятий, предотвращающих утечку закрытой информации.

	Контроль за деятельностью по обеспечению сохранности государственных секретов, соблюдению требований установленного режима секретности осуществляется органами госбезопасности. (ФСБ)

	Уголовная ответственность лиц, виновных в разглашении секретных сведений, в утрате секретных документов и изделий.

	Режим секретности включает в себя
	\begin{enumerate}
	\item	Порядок установления степень секретности сведений, содержащихся в работах, документах и изделиях.
	\item	Порядок допуска граждан к работам, документам и изделиям.
	\item	Порядок выполнения должностными лицами своих \textbf{должностных} обязанностей, связанных с сохранением госудрсственных и служебных тайн, по соблюдению режима секретности.
	\item	Порядок обеспечения секретности при проведении на предприятии работ  закрытого характера.
	\item	Порядок обеспечения секретности при ведении секретного делопроизводства.
	\item	Порядок обеспечения секретности при использовании технических средств, передаче, обработке и хранении информации. (пользование связью, множительным оборудованием итп).
	\item	Порядок обеспечения секретности при осуществлении предприятиями, где ведутся секретные работы контактов с зарубежными фирмами.
	\item	Порядок проведения служебных расследований по фактам разглашения секретных сведений.
	\end{enumerate}






	Структура предприятия
	Глава"исследование организационной структуры предприятия"
	Подразделения, функции (отделов и kbhtrnjhjd(?))
	Схема по Жукову (VISIO). Подписать снизу рис. , точка не ставится.



	\subsection{Порядок засекречивания и рассекречивания сведений, документов и продукции}

	Засекречивание сведений - введение ограничений на их распространение и на доступ к их носителям.
	При засекречивании сведений руководствуются следующими положениями
	\begin{itemize}
	\item	Решение проблем засекречивания решается с позиции государственной значимости этих сведений, причём стоит учитывать противоречивость двух тенденций: стремление обеспечит надёжность сохранности, и недопустимость необоснованного массового засекречивания.
	\item	Объективный характер обеспечения степени секретности сведений.
	\item	Оптимизация объёма засекречиваемых сведений.
	\item	Периодический просмотр степени секретности сведений на предмет снятия или снижения грифа секретности (рассекречивание).
	\end{itemize}

	Принципы засекречивания
	\begin{itemize}
	\item	Законность засекречивания
	\item	Обоснованность - установление целесообразности засекречивания сведений.
	\item	Своевременность - отнесение сведений к гостайне и засекречивание с момента их получения или заблаговременно.
	\end{itemize}

	Перечень сведений, составляющих гостайну. Три перечня. (сведений, составляющих гостайну, открытый, в законе);(сведений, отнесённых к гостайне, открытый, с указанием ведомственной принадлежности);(перечень с классами секретности, закрытый.)
	\begin{itemize}
	\item	Сведения военной области
	\item	Сведения в области экономике, науки и техники
	\item	Сведения о внешней политике и экономике
	\item 	Сведения о разведывательной, контразведывательной, следственной деятельномти и противодействию террористам.
	\end{itemize}

	Степени секретности и грифы носителей. (Секретно, совсекретно, особой важности (высшая))
	Особой важности - ущерб государству в целом.
	Совершенно секретно - ущерб министерствам, ведомствам или отрасли экономики.
	Секретно - всё не вошедшее - ущерб предприятию.

	Законом засекречивание сведений, не составляющих гостайну недопустимо.


	Реквизиты носителей, составляющих гостайну - инф-ия, проставляема на носителе либо сопроводительной документации.
		\begin{itemize}
		\item 	Степень секретности
		\item 	Ссылка на перечень сведений, подлежащих засекречиванию
		\item 	Сведения об органе госвласти или предприятии, засекретившей носитель (рассекречивает только засекретивший)
		\item 	Регистрационный номер
		\item 	Дата засекречивания
		\item 	Условия рассекречивания (дата рассекречивания)
		\end{itemize}

	\subsubsection{Порядок и правила отнечения документов к различным степеням секретности}
	Засекречивание проводится в соответствии с перечнем из закона о гостайне.
	Правом отнесения к гостайне обладают руковолители органов государственной власти в соответствии с перечнем должностных лиц, наделённых полномочиями по отнесению сведений к гостайне (второй открытый перечень).

	Для осуществления единой политики в областизаасекречивания создаётся межведомственная комиссия п защите гостайны. (формирует перечень свелений, отнесённых к гостайне) (второй перечень)

	Органами госвласти, руководители которых имеют право отнесения к гостайне разрабатывается развёрнутые перечни сведений, подлежащих засекресчиванию (утверждается руководителем органа власти с соответствующими полномочиями (министр обороны, напр.))

	В области военной техники могут разрабатываться отдельные перечни. (утверждаются руководителями органов власти, которые участвуют в данных работах)
	Этот перечень содержит степень секретности. Иногда засекречиваются сами перечни.

	Перечни сведений, подлежащих засекречиваются до органов власти по их вопросу полностью, либо в части изх касающейся; до подчиненных предприятий в части, их касающейся; до предприятий, участвующих в проведении совместных работ.

	Основанием для засекречивания является соответствие сведений перечню.


	\subsubsection{Порядок разработки перечня}

	\subsubsection{Процедура предварительного засекречивания}

\end{document}