\documentclass[a4paper,12pt]{report}
\usepackage{../GS6}

\begin{document}

	\def \nocredits {}
	\def \LineE {Конспект по дисциплине}
	\def \LineF {Информационные технологии}

	\maketitle

%10.03.15

\subsection*{Донецкая Юлия Валерьевна}


	Отчётность  - лабы, моделирование (выбрать область, написать "тз".
	\begin{itemize}	
	\item	ТЗ	
	\item	Модель в нотации IDEF0
	\item	Диаграммы UML - (диаграмма деятельности, диаграмма состояния)
	\item	(очень опционально - алгоритмы). Описывать можно со стороны реверса.
	\item	Выводы
	\end{itemize}
	
	Сроки - к зачётной неделе.
	
	След занятие - тексты заданий
	Цель описания, цель построения модели
	Входные данные, цель на выходе (с т.з. информации)

	Моделирование иб-процессов:
	\begin{itemize}
	\item	Описание информационных потоков (as-is, to-be)
	\item	Описание угроз, нарушителей, ...
	\item	Средства представления (методы защиты)
	\item	Методы устранения и предотвращение
	\end{itemize}


	\subsection{IDEF0}
	(стандарт IDEFX) Распространённые варианты 0,3 (с порядком действий).
	3 обязательных составляющих любой диаграммы:
	\begin{itemize}
	\item	Цель построения модели
	\item	Указание степени детализации
	\item	Точка зрения (коллектива)
	\end{itemize}
	
	"правило хорошего тона IDEF0" 
	Любое описываемое действие изображено прямоугольником, с входными данными, выходными данными, механизмы (сотрудники, системы, ...), управление (нормативные акты, ГОСТ, ФЗ, ...).
	In, out - существительные. Действие - глагольная форма.
	
	Next lec - моделирование, 
















\end{document}	
	