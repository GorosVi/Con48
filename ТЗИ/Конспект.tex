\documentclass[a4paper,12pt]{report}
\usepackage{../GS6}
\sloppy
\begin{document}

	\def \nocredits {}
	\def \LineE {Конспект по дисциплине}
	\def \LineF {WEB-технология}

	\maketitle


Пт 930 - донецкая неч => без пар до 12
числа
%10.03.15

\subsection*{Заикин Константин Ниолаевич}

Технические средства охраны

	+ курсач

	>> Печатно, times 14, гост 7.32 (рекомендательно)

	Методичка для выполнения курсовой (в электронном виде)

	Суть: исходно план помещений. (не менее шести)

	Двери - рекомендательно (ближе) по пути эвакуации
	Материалы предоставлять в печатном виде.
	Формулы набирать в реакторе, без транслитерации знаков. Вычисления - в развёрнутом виде, с размерностями.
	Двери - по умолчанию - с доводчиками.

	Приёмно-контрольный прибор (панель у двери)

	Без мебели


	Ошибки:
	Недостаток, или излишние ТС
	Арифметика =)
	Залание надо носить на все пары
	Недостаток характеристик прибора в сводных таблицах
	Неправильное применение прибора



	%Lec<02.09.15>
	\subsection{Современная концепция безопасности объекта}
	Соверменная концепция состоит из этапов:
	\begin{itemize}
	\item	Устанавливаем, что защищать
		\item	Характер деятельности
		\item 	Занимаемая территория и окружающая обстановка
		\item	Организационная структура предприятия ( => структура информационных потоков, внутренних и внешних, закрытых и открытых)
	\item 	Список возможных угроз с указанием источника и вероятности -модель угроз-
	\item 	Выбор методов из средств противостояния (политика безопасности предприятия(свод документов и правил в области безопасности в целом и в области информационной безопасности в частности))
	\end{itemize}

	Угрозы: жизни и здоровью, среде обитания, имуществу и информации.

	Определение Системы Безопасности -СБ -это совокупность методов и средств поддержания безопасного состояния, предотвращения, обнаружения и ликвидации угроз жизни, здоровью, среде обитания, имуществу и информации.

	Система безопасности - интегрированная и комплексная
		Интегрированная: компоненты объединены, и работают, используя общие возможности.
		Комплексная - система, покрывающая задачу, не обязательно связная.

	СБ представляет совокупность технического, ресурсного и правового обеспечения.
	Техническое обеспечение - аппаратные и программные компоненты.
	Ресурсное - финансы, материально-техническое обеспечения и трудовые ресурсы.
	Правовое - законодательная и нормативная база службы безопасности.



	Основные технические средства (фотка)
	Технические средство охраны - (гост) ТСО - конструктвно законченное, выполняющее самостоятельные функции устройство, входящее в состав охранной, Тревожной сигнализацией, контроля и управления доступом, охранного ТВ, освещения, оповещения и других систем для охраны объекта.

	Технически средства обеспечения безопасности
	\begin{itemize}
	\item	Средства защиты информации
	\item	Средства инженерной защиты информации <<не будет>>
	\item	Средства видеонаблюдения
	\item	ТС пожарной сигнализации
	\item	ТС охранной сигнализации
	\item	Средства связи
	\item	СКУД
	\item	Оружие и спецсредства защиты
	\end{itemize}


	\subsection{Основные принципы построения системы безопасности.	}
	\begin{itemize}
	\item	Зональный принцип
	\item	Принцип равнопрочности
	\item	Надёжности и живучести
	\item	Регулярности контроля функционирования
	\item	Адаптивности
	\item	Адекватности
	\end{itemize}

	Зональный принцип : категорирование помещений по уровню секретности, и группировка в адекватном порядке.
	Равнопрочность: СЗ должна иметь одинаковый уровень защиты вне зависимости от маршрута злоумышленника.
	Надёжности и живучести: живучесть должна быть на адекватном уровне для функционирования при повреждении жизненно важных узлов либо питания и связи.
	Регулярности и контроля функционирования - наличие встроенных средств самодиагностики и тестирования.
	Адаптивности: система должна быть достаточно гибкой и предусматривать возможность доработки при изменении характера угроз.
	Адекватности: принятые меры и ТС должны соответствовать оговоренным угрозам и модели нарушителя


	\subsection{Охранные извещатели. Классификация.}

	Извещатель - устройство для формирования извещения о тревоге при проникновении или попытке проникновения, или для инициирования сигнала тревоги потребителем.

	Классификация:
	По типу обнаруживаемых тревожных событий
	\begin{itemize}
	\item	Движение (объектов в контролируемых зонах)
	\item	Разрушения (стёкол, стен)
	\item	Изменения положения контролируемых объектов (открывание двери, смещения предметов)
	\item	Пересечения контролируемой зоны
	\item	Присутствия объекта в некотором пространстве
	\item	Изменения параметров среды (наличия газа, дыма)
	\end{itemize}

	По используемым физическим принципам:
	\begin{itemize}
	\item	ИК
	\item	Радиоволновые
	\item	Ультразвуковые
	\item	Акустические
	\item	Пьезоэлектрические
	\item	Емкостные
	\item	Индуктивные
	\item	Электростатические
	\item	Вибрационные
	\item	Температурные
	\item	Оптические
	\item	Контактные
	\end{itemize}

	Активные и пассивные

	Конструктивно извещатели состоят из датчиков и детекторов. Датчик контролирует определённый параметр среды, детектор - контролирует и обрабатывает, извещатель формирует сигнал исходя из входных значением детектора. В ОС - извещатель.

	Извещатели подразделяются по способу электропитания - проводные(неавтономные), автономные(беспроводные.
	Шлейф сигнализации - электрическая цепь, соединяющая выходные цепи охранных (пожарных) извещателей, включающая в себя вспомогательные (выносные) элементы (диоды, резисторы итп) и соединительные провода и предназначенная для выдачи на приёмно-контрольный прибор извещений о проникновении (попытке проникновения), пожаре и неисправности, а в некоторых случаях и для подачи электропитания на извещатели.

	Беспроводные извещатеи включены вне зависимости от включения


	По способу идентификации извещателя:
	\begin{itemize}
	\item	Безадресные - пожарные системы, конкретный извещатель неизвестен.
	\item	Адресные - передаётся код извещателя.
	\end{itemize}

	По конструктивному исполнению:
	\begin{itemize}
	\item	Внутреннего исполнения - в помещениях
	\item	Внешнего исполнения - -50%+50, антивандальность.
	\end{itemize}

	\begin{itemize}
	\item	Комбинированные - для повышения вероятности определения тревоги используется несколько каналов с ранми физическими принципами.
	\item	Совмещённые - один прибор выдаёт два различных события (движения и разбития стекла)
	\item	Ручные извещатели (пожарные)
	\end{itemize}

	\subsection{Магнитно-контактные извещатели (3)}
	Простота, дешевизна и надёжностью

	Геркон+резистор+магнит
	Рабочий зазор - 2-
	Коммутационный ток геркона - до 30-300 мА
	Способ монтажа - заглубленный и внешний. На деревянной дверях применять накладные извещатели не допускается.
	Устойчивы к вибрациям и нагрузкам.



	Особенности установки
		На плотно закрываемые двери с доводчиками, порядка 20 см от внешнего косяка двери.
		На деревянных дверях - за металлической пластиной.


	Одним герконом защитить помещение невозможно.



	\subsubsection{Оптико-электронные извещатели (пассивные)}

	Класс приборов - извещатели движения, реагирующие на изменения уровня ИК-излучения в результате перемещения человека в зоне обнаружения, вне зависимости от пути проникновения в помещение.

	Могут контролировать как объём или коридор, или как узкую полосу.
	Основные физические принципы:
	\begin{itemize}
	\item	Излучение тел, нагретых выше температуры абсолютного нуля. Излучение тел, нагретых ниже $200^{\circ} C $ находится в ИК-диапазоне.
	\item	Пироэлектрический эффект - изменение поляризации диэлектрика при воздействии теплового излучения, и как следствие, появление эл. зарядов н гранях пироприёмника, которые могут быть зарегистрированы.
	\end{itemize}
	Теорема Вина: $\lambda = \frac{2899}{T_{[K]}} [\mbox{мкм}]$.

	Параметры:
	\begin{itemize}
	\item	Чувствительность $[\mbox{В}/\mbox{Вт}]$
	\item	Уровень собственных шумов
	\item	Эквивалентна мощность шума и обнаружительная способность - минимальная мощность излучения, когда выходной сигнал равен уровню собственного шума.
	\item	Оптические свойства - полоса пропускания оптического фильтра, угол обзора, производительность в пределах угла обзора.
	\end{itemize}

	Полоса пропускания рассчитана на детекцию человека - диапазон 8-14 мкм.

	Стандартная цель - конструктивный элемент, характеристика излучения в ИК-диапазоне аналогичны характеристикам человека(1500x300x235 мм) распределение температуры поверхности - в пределах 2-3 гр. Стандартный фон - 20-25 гр.

	Сегментированная ДН (фото)

	Зона обнаружения - зона, в которой извещатель выдаёт извещение тревоги о перемещении стандартной цели на постоянном расстоянии от извещателя, и зона обнаружения состоит из элементарных чувствительных зон оптической ДН извещателя. Для повышения чувствительности слои диаграммы направленнности провёрнуты на некоторй угол.

	Скорость перемещения - от 0.3 - 3 м/с.

	Угол обзора зоны обнаружения - угол, заключённый между двумя условными прямыми, исходящими от извещателя и являющихся границами зоны обнаружения.
	Наиболее распространённый угол обзора - $90^\circ$

	Оптимально для дверей - установка под $90^\circ$ к пути входа злоумышленника.


	Основные типы диаграмм направленности
	\begin{itemize}
	\item	Самая распространённая - широкий угол (объёмная)
	\item	Коридорная (линейная) - узкий угол, $10^\circ$
	\item	Вертикальная занавеска - угол уже, порядка одной зоны
	\item	Горизонтальная занавеска - узкая плоскость
	\item	Штора
	\item	Круговая - офисы с перегородками
	\end{itemize}

	Корректировка - наклейками по внутренней поверхности линзы.

	Оптические системы - зеркальные и линзы Френеля.

	{Помехи}
		Экраны от насекомых оптической камеры

		Вторичная стандартная цель - элемент, имеющий в ИК-спектре характеристики небольшого животного (типа мышь, 300dx150 мм)


		// ====================================



	... nnext0b ещ ьуку

	Воторичная цель.


	Методы, позволяющие устранить помехи от животных и насекомых на объекте:

	\subsubsection{Пространственная селекция}
	Применяется "аллея для животных" - горизонтальная занавеска, либо понижение чувствительности в определённых зонах (где ожидается появление животных).

	\subsection{Параметрическтая селекция}
	Кроме основных извещателей, необходимых для контроля помещения, используется дополнительный извещатель - и срабатываем по сумме.

	\subsubsection{Последовательные алгоритмы}
	Устанавливается характеристики движения объекта (количество ног. характер движения)

	\subsection{Установка и регулировка PIR-извещателей на объекте}

	Установить необходимо по возможности поперёк направления движения потенциального нарушителя. (окна, двери)
	Не направлять на потенциальные источники ложных тревог (окна - засветка, отопительные приборы - восходящие воздушные потоки, мебель не должна загораживать помещение)
	Монтаж обычно производится на высоте 2.4 метра, обычно крепятся под углом в 45гр.к поверхности.
	Регулировки:
	\begin{itemize}
	\item	Дальность
	\item	Чувствительность
	\item	Антиблокировка - от закрашивания - нижнее окно. Решается активным ИК-каналом.
	\item	Защита от вскрытия корпуса (тампер)
	\item	Включение (выключение) светодиода тревоги - необходимо отключить.
	\end{itemize}


	\section{Радиоволновые извещатели}

	В основе действия прибора - эффект Доплера. $df = 2\frac{V_p}{\lambda}$; $\lambda = V_c / f_с$.

	$f_c = 10GHz; V_p = 0.1M/s df_1 = 6Hz; V_p = 3M/s df_2 = 200Hz$.

	Могут срабатывать от люминесцентных ламп, вентиляторов.
	Диапазоны частот -
	\begin{itemize}
	\item	S - 2.4 GHz
	\item	C - 5.4 GHz
	\item	X - 10 GHz
	\item	K - 24 GHz
	\end{itemize}

	Чем выше частота - тем больше доплеровский сдвиг - хорошо.
	Чем выше частота - тем меньше антенна.
	Чем ниже частота - тем выше проникающая способность через перегородки (хуже) - X и K практически полностью поглощаются перегородками.

	Высокочастотные приборы значительно дороже.
	Основные требования:
	\begin{enumerate}
	\item	Максимальная эффективность - движение в направлении к прибору (по радиусу)
	\item	Для уменьшения количества лохных срабатываний не направлять на окна, стеклянные двери.
	\item	В помещении не должно быть сквозняков (а след-но колебаний занавесок и т.п.).
	\item	Не включать люминесцентные лампы.
	\item	Устанавливать на капитальных стенах (колебание основы может вызвать ложное срабатывание)
	\end{enumerate}

	\section{Извещатели разбития стекла}
	В зависимости от типа стёкол на объекте зависит выбор извещателя.
	\begin{enumerate}
	\item	Листовое стекло
	\item	Закалённое стекло - листовое стекло, при разбитии рассыпается мелкими фрагментами.
	\item	Многослойное стекло - слои стекла с плёнкой. Разрушается, не рассыпаясь.
	\item	Стекло, покрытое плёнкой -не разрушается от ударов
	\item	Армированное стекло
	\item	Стеклопакет
	\end{enumerate}

	Физические принципы:
	\begin{enumerate}
	\item	Нарушение целостности стекла
	\item	Механические колебания стекла
	\item	Акустические колебания в пространстве
	\end{enumerate}

	Нарушение целостности стекла: контролируется электроконтактным способом. Плёнка может рваться сама. Единственный способ, работающий в условиях тряски (поезд итп.)
	Возможно подключение к сетке армированного стекла.

	Механические колебания стекла - ударно-контактные извещатели - пьезоэлектрические (активные и пассивные)

	Пассивные пьезоизвещатели дешевле, но могут производить ложные срабатывания.
	Активные - производят акустические колебания. Два канала - один на отклеивание, один на колебания.


	Акустические колебания стекла. Витражи.

	Регистрируют частоты диапазона разрушения стекла.

	Диаграмма направленности извещателя - обычно сфера. Дальность - 7-10 м.

	Правила установки:
	\begin{enumerate}
	\item	Акустические ИРС должны быть установлены в пределах зоны действия. Допустимое незкрытие - 10 см. (даже в глубоких проёмах).
	\item	Высота установки - 1.8 м.
	\item	По возможности устанавливать в прямой видимости стёкол.
	\item	Могут срабатывать через лёгкие шторы/тюль. Тяжёлые шторы требуют установки извещателя внутри проёма.
	\end{enumerate}





	\subsection{Пожарные извещатели}

	Извещатели предназначены для первичного обнаружени физических факторов, сопутствующих пожару, таких как: тепло. дым ,открытое пламя и переаче тревожных извещений по шлейфу на приёмно-контрольный прибор.
	Классификация по способу обнаружения
	\begin{itemize}
	\item 	Тепловые (максимальные - по достижению предельной температуры, либо дифференциальные - по предельной скорости изменения темперауры, либо максимально-дифференциальные.)
	\item 	Дымовые (Точечные (зона - конус) и линейные (термокабель, )
	\item 	Световые
	\item 	Газовые
	\item 	Ионизационные
	\item 	Ручные
	\end{itemize}

	Световые извещатели реагируют на ИК, УФ или видимое излучение открытого пламени пожара.
	Газовые - на газы, выделяемые при тлении или горении (наиболее дорогие и качественные).


	Деление на адресные и безадресные.
	Согласно ГОСТ - если используются безадресные извещатели, то каждая точка внутри помещения должна контролироваться двумя извещателями (можно обязательными).

	Расстановка пожарных извещателей:
	Расстояние между соседними извещателями - максимум 9 м. В коридорах - можно ставить в 1.5 раза реже.
	Шлейф пожарных извещателей идёт отдельно от охранных. (но можно с охранкой одной трассой, и даже в одном кабеле)

	Контроль фальшпотолков и фальшполов (оптоэлектронные извещатели, точеные не канают)


	Ручные извещатели (на путях эвакуации, на расстоянии максимум 50 м друг от друга)



	\section{Расчёт минимального сечения токоподводящей жилы шлейфа сигнализации}
	Шлейф - электрическая цепь, соединяющая выходные цепи охранных, пожанных или охранопожарных извещателей, включающая в себя вспомогательные элементы (дтоды? резистоы, соединительные провода), и преназначенные для выдачи на ПКП сообщений о проникновении либо попытке проникновения, пожаре или неисправности, в некоторых случаях и для поданния электропитания на извещатели.

	Кабели
	\begin{itemize}
	\item	Кабель - несколько жил в оболочке
	\item	Провод - несколько жил в оболочке (может идти голым)
	\item	Шнур - более гибкий кабель, сечением до $1.5 \mbox{мм}^2$
	\end{itemize}

	Критерии выбора кабеля
	\begin{itemize}
	\item	Механическая прочность
	\item	Допустимый нагрев кабеля (максимальной силой тока)
	\item	Допустимое падение напряжения (в курсаяе - исходя из расчёта идзвещатель - своя пара проводов)
	\item	Условия эксплуатации
	\end{itemize}

	Материал токопровоящей жилы кабелей.

	Медь: классы гибкости (больше примесей - гибче провод - выше сопротивление)
	Сопротивление принимаем 0.02 ом/мм2*м

	Зависимоить сопротивления жилы от нагрева: $$\rho =  \rho_0(1+\alpha(t - 20)) $$. t - градусы Цельсия.
	Для Cu и Al $\alpha = 0.004$ (б/разм.).

	Падение напряжения
	$\Delta V_{\mbox{кр} = 5\% = 0.5V = \mbox{логически}}$.

	С точки зрения ГОСТ нормируется удельное сопротивление кабеля.

	Сечение токоподводящей жилы не может быть меньше 0.5 мм2 (нормы пожарной сигнализации)

	Длина цепи к извещателю - 2l.


	\subsection{Вторичные источники питания}

	Нормы качества электроснабжения (ГОСТ) - 220V +- 5\%, кратковременно - в пределах 10\%, по частоте - +- 0.2 Hz, кратковременно в пределах 0.4Hz

	Регламент "ПУЭ" (главный руководящий документ)

	Категории нагрузки по ПУЭ - 1, 2, остальные категории.
	1 - наиболее ответственные (должны питаться от независимых источников питания), с активным АВР. (органы власти, )
	2 - менее ответственные (от 2-х независимых источников, время АВР больше)


	Деление источников вторичного питания: источники резервного питания и источники бесперебойного питания.



\end{document}