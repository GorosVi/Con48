\documentclass[a4paper,12pt]{report}
\usepackage{../GS6}

\begin{document}

	\def \nocredits {}
	\def \LineE {Конспект по дисциплине}
	\def \LineF {Безопасность жизнедеятельности}

	\maketitle


%03.09.15
% -5 мин.
\subsection*{Слободянюк Александр Александрович}

Фотки =>

Литература => фотки >> долин желателен (для защиты лаб)

	БЖД - система знаний, направленных на обеспечение безопасности, сохранения жизни и здоровья человека в производственной и непроизводственной среде с учётом влияния человека на среду обитания.
	БЖД, т.о. является комплексной наукой, изучающей общие закономерности опасных явленийи соответствующие методы исредства защиты человекпа, при которой \textbf{с определённой вероятностью} исклчаются потерциальные опасности. влияяющие на жизнь и здоровье человека, его потомтво.
	
	Жизнедеятельность - повседневная деятельность человека (втч и отдых), способ существования
	
	
	Безопасность - комплексная система мер по защите человека  и среды обитания, определяемых его конкретной деятельностью.
	
	Охрана труда - система сохранения жизни и здоровья работников в процессе жизнедеятельности ,вклюючающая в себя
	
	
	ОТ включает в себя разделы:
	\begin{itemize}
	\item	Законодательные акты
	\item	Управление охраной труда
	\item	Организация охраны труда
	\item	Промышленная санитария
	\item	Пожарная безопасность
	\item	Промышленная экология
	\end{itemize}
	\sim
	
	Основная цель БЖД - защита человека в техносфере от негативноо воздействия антропогенного и естественного происхождения, и достижения комфортных условий жизнедеятельности
	
	Средством достижения этой цели является реализация обществомзнаний и умений, направленных на уменьшение в техносфере 
	
	
	Структура БЖД включает в себя общественные знания, медико-биологическия знания, технолого-технические знания. законы о природных явлениях

	Главной задачей БЖД является анализ источников и причин возникновения опасностей, прогнозирования и оценка их воздействия во времени и пространстве.
	БЖД решает 3 взаимосвязанные задачи: 
	\begin{itemize}
	\item	Идентификация опасности
	\item	Защита от опасностей, с учётом затрат и выгод
	\item	Ликвидация возможныхх опасностей
	\end{itemize}
	
	БЖД - система знаний направленных на обеспечение безопасности через разработку методов и средств обеспечения безопасности в производственнй и непроизводственнй среде с учётом влияния человека на среду обитания
	
	ЦелиЖ
	\item	Уменьшение вероятноти проявоения опасностей
	\item	Прогнозирование чрезвычайных ситуаций
	Обеспечение готовноти к возможным стихийным бедствиям, аварияям и катастрофам.
	
	В центре внимания БЖД - человек.
	
	
	
	Безопасность - состояние защищённости жизненно важных интересов общества и государства от внутренних и внешних угроз
	Жизненно важные интересы - потредбности, удлвлнворение обеспечивает существование и возможность развития личности, общества и государстваю
	БЖД - состояние деятельности, при которой с определённой вероятностью исключено проявление опасности.
	Понятие безопасность - сложное, и носит системный характер. Все системы и подсистемы испытывают взаимовлияни.
	
	Национальная безопасность Ж
	Внутренняя/ внешняя
	По сферам общественной жизни и человеческой деятельнояти
	По объектам: личности, общества и государства
	
	
	Опасность - явления, процессы или объекты, спообные в опредделённых усолвиях вызывать неделательные последствия, наносить ущерб здоровью, или угрожать жизни.
	Оющие характеристики опасностей:
	\begin{itemize}
	\item	Носят вероятностный характер
	\item	ПОтенциальность/ скрытность
	\item	Перманентность
	\item	Тотальность
	\end{itemize}
	
	Классификация опасностей:
	\item	ПО характеру воздействия
		\item	Мезанические
		\item	Химические
		\item	биологическте
	\item	По вызываемым последствиям
		\item	заболевания
		\item	утопления
		итд
	\item	По приносимому ущербу
		\item	социальные
		техническик
		\item	экологические 
		\item	экономические
	\item	По структуре
		\item	простые
		\item	производные
	\item	По локализации воздействия
		\item	Литосфера
		\item	Гидросфера
		\item	атмосфера
		\item	космос
	\item	По сфере проявления
		\item	Бытовые
		\item	спортивные
		орожно-транспортные
		производственные
	\item	По времени проявления
		\item	импульсные
		\item	кумулятивные
	\item	По реализуемыой энергии
		\item	активные
		\item	пассивные
		
		
	ГОСТ 12.0.003-99
	Возможные опасности делятся на опасные и вредные факторы:
		Опасный фактор - может привестик травме или резкому ухушению здоровья. Регламентируется правилами ТБ		
		Вредный фактор - модет привести к ухудшению самочувствия, повышенной утоляемости, заболеваниям - шум. вибрация, ЭМИ. регламентируется производственной санитарией.
		
		Некоторые факторы могут проходить превращения полезный/вредный/опасный.
		
		
		НА человека, занятого процессом труда, среда воздейстует комплексом факторов.
		Комплекс фактров, хар-х сам труд - психофизиологические факторы
		Комплекс внешних условий или факторов - санитарно-гигиеническте условтя труда.
		
	Опасности:
	\begin{itemize}
	\item	Физические
		\item	Подвихные механизмы
		\item	Загазованность, пыль
		\item	Холод
		\item	Вибрации
		\item	Слабая освещённость 
		\item	ЭМИ
	\item	Химические
		\item	Токсические
		\item	Раздражающие
		\item	мутагенные 
		\item	Канцерогенные
		По пути проникновения: 
		\item	органы дыхания
		\item	Желудок
		\item	слизистые
		\item	кожные покровы
	\item	Биологические
		\item	бактерии
		\item	вирусы
		\item	спирохеты
		\item	грибки
		\item	растения
		\item	животные
	\item	Психофизиологические
		\item	Физ нагрузки (динамические, статические, нервно-психическте)
		\item	Умственые (монотонность, умственные перенапряжения, эмоциональные перегрузки)
	\end{itemize}
	Общие токсические вещества - отравление всего организма - Hg, Pb, CO
	
	пдк - предельно допустимые концентрации вредного вещества.
	Вредные вещества (гостделят на классы (1- чрезвычайно опасные; высокоопасные; умеренно опасные; 4 малоопасные)
	 Критерии опасноти устанавливаются в зависимости от величины пдк, средней смертельной дозы, дозы острого воздействия
	 
	Аксиомы:
	\begin{itemize}
	\item	Любые объекты. процессы потенциально опасны
	\item	Любая (без)деятельность потенциально опасна для человека
	\item	Ни в одном виде деятельностьи нельзя добиться абсолютной безопасности.
	\item	Безопасность системы можно достигнуть с любой степенью внроятности, не исключающей существование системы
	\end{itemize}
	
	\subsection{Теория риска}
	Риск - вероятность реализации негативного воздействия в зоне пребывания человека, и количественная оценка опасности. Безразмерна величина в интервале [0..1]
	Значение риска получаются из стстистики несчастных случае
	При использовании статданых формуоа выглядит $R = (N_{\mbox{чс}}/N_o) \le R_{\mbox{доп}}$
	
	Ноксосфера - зона формирования опасностей 
	Гомосфера - зона деятельности человека
	Опасности локализуются на их пересечении
	
	Индивидуальный риск
	Коллективный риск - травмирование двух и более человек
	Риск исползуется при сравнении безопасности различных отраслей.
	
	Приемлимый риск - уровень смертности, травматизма или инвалидности, не влияющий на экономические показатели предприятия/отрасли/государства.
	
	
	График суммарного риска имеет минимум при разумном соотношении между затратаи на тесферу. - приемлимый риск.
	Общим приемлимым риском считается показатель $10^{-6}$ в год
	
	
	
	Методические подходы к определению риска
	\begin{itemize}
	\item	Инженерный (статистика. вер. анализ)
	\item	Модельный (модели воздействия)
	\item	Экспертный
	\item	Социологический (опрос населения)
	\end{itemize}
	
	Пути уменьшения риска
	
	Уменьшить риск можео следующими методами
	\begin{itemize}
	\item	Совершенствование технологии
	\item	Срадства защиты
	\item	Подготовка и обучение персонала
	\item Организационные методы
	\item	Страхование, компенсации
	\end{itemize}
	
	Метод -  способ достижения цели, из знания общих закономерностей
	
	Принципы обеспечения безопасности делят на ориентирующие, технические, организационные, управленческие
	
	
	Ориентирующие - идеи, определяющие направления поиска.
	Технические - непосредственное предотвращение опасности
		Изоляция
		Экранирование
		Принцип слабого звена - расчётное ослабление элементов конструкции - УЗО, итп.
		Принцип прочности - упрочнение остального.
	Организационные
		Надзор
		КОнтроль
		Защита временем - перерывы, сокращённый рабочий день
		Пригцип информации - предупредительные надписи, маркировкиы
	Управенческие
		Плановости
		Стимулирования
		Компенсации
		эффективности
	
	Обеспечение безопасности
	Метод А: пространственное или временное разделение гомосферы и ноксосферы (дист управление, автоматизация)
	Метод Б: Нормализация ноксосферы ттехническими средствами  - шумопылезащита
	Метод В: Адаптация человека: профотбор, обучение, сиз
	
	
	Виды систем безопасности
		Система личной и коллективной ьезопасности
		Охраны природной среды
		Госбез
		глобальной безопасности
		
		
	Управление безопасностью жизнедеятельности - воздействие на систему человек-среда в неправлении уменьшения опасности
	
	
	Действие теплового излучение зимеряется через интенсивность теплового облучения и измеряется в $вт/м^2$
	Количество теппла приттеплообмене излучением зависит от температуры окружающих поверхностий и темепратуры тела человека, те отдачатепла происходит при температуре тела больше т окр поверхностейй
	
		
		
	
>>> Практика - допуск на зачёт - со сданной практикой
	Лабы 6 + 2 типовика, из методички 
	
	Типовик иск освещение. заземление		